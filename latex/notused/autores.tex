\section{Autores}
\subsection*{Christina von Flach Garcia Chavez}
    \begin{itemize}
        \item E-mail para contato: flach@ufba.br
%        \item Número de associado: 8925
        \item Instituição: Universidade Federal da Bahia (UFBA)
        \item Mini-bio: Christina von Flach é Professora Associada do Instituto de Computação da UFBA. Seus interesses de pesquisa incluem aspectos sócio-técnicos de ecossistemas de software, sustentabilidade do software, software livre e educação em engenharia de software. Co-organizou o I Workshop sobre Práticas de Ciência Aberta na pesquisa em Engenharia de Software, realizado durante o CBSOFT'21. Coordena os \textit{Seminários sobre Desafios e Práticas da Ciência Aberta na Computação} em parceria com a Prof. Claudia Maria Bauzer de Medeiros, evento-satélite do CSBC 2023. Lecionou cursos de pós-graduação com um subconjunto da temática deste curso em quatro ocasiões. Orientou a primeira dissertação de mestrado no Brasil sobre o tema deste curso, ``Sustentabilidade técnica de software acadêmico no domínio de ferramentas de análise estática'', defendida por um dos co-autores. Atualmente, é orientadora da pesquisa de mestrado de Daniela Feitosa (co-autora) sobre este tema.
        \end{itemize}

\subsection*{Joenio Marques da Costa}
    \begin{itemize}
        \item E-mail para contato: joenio@joenio.me
        \item Instituição: Université Gustave Eiffel (UGE)
        \item Mini-bio: Joenio M. Costa é engenheiro de \textit{software de pesquisa} (Research Software Engineer) no laboratorio LISIS (Laboratoire Interdisciplinaire Sciences Innovations Sociétés), trabalhando como desenvolvedor \textit{backend} nos projetos RISIS (Research Infrastructure for Science, technology and Innovation policy Studies) e CorTexT. Seus interesses de pesquisas incluem sustentabilidade de software, software de pesquisa, citação de software e qualidade de software. É colaborador do sistema operacional livre Debian, instrutor de Computação no \textit{The Carpentries} e embaixador do \textit{Sofware Heritage}, o arquivo universal de preservação de código fonte de software.
    \end{itemize}

\subsection*{Daniela Soares Feitosa}
    \begin{itemize}
        \item E-mail para contato: danielafeitosa@gmail.com
        \item Instituição: Universidade Federal da Bahia (UFBA)
        \item Mini-bio: Daniela Feitosa é engenheira de software e estudante de mestrado
        no Programa de Pós-graduação em Ciência da Computação (PGCOMP) na Universidade Federal da Bahia (UFBA). Seus interesses de pesquisa incluem sustentabilidade do software, software livre, qualidade de software e educação em engenharia de software. Seu projeto de pesquisa no Mestrado está alinhado com os objetivos deste curso.    
    \end{itemize}