

\subsection{Preservação de Software}

\textit{Software Heritage}\footnote{\url{https://www.softwareheritage.org}}, que tem como objetivo coletar e preservar software em formato de código-fonte para preservação e acesso público, além de propostas para o uso de identificadores persistentes específicos para software, como o SWHID\footnote{\url{https://www.swhid.org}}, que podem facilitar a referência e o rastreamento de \textit{Software de Pesquisa}.

Preservar codigo-fonte de software se justifica pela relevancia que o software representa na sociadade moderna, sendo considerado em muitos casos uma importante heranca cultural, no contexto academico codigo-fonte do software de pesquisa contem conhecimento cientifico codificado em linguagem de programacao, este conhecimento pode ser aplicado em inumeros contextos mas se o codigo-fonte nao eh preservado ele se perde.

Ja houve casos onde repositorios de codigo-fonte foram descontinuados, entre os exemplos podemos citar o Gitorious.org e o Google Code que em 2015 foram descontinuados e os projetos hospedados em tais plataformas tiveram seu codigo-fonte deletados, em   2020 o BitBucket anunciou a remocao de mais 250.000 repositorios, em 2022 o GitLab.com comecou a considerar remover projetos inativos por mais de 1 ano ou mais. Em todos estes casos inumeros repositorios de codigo-fonte sao perdidos e o projeto Software Heritage (SWH) surge como proposta de garantia que este risco nao se repita, uma vez que o SWH coleta e preserva de maneira permanente projetos hospedados em repositorios publicos e abertos.