%-------------------------------------------%
\section{Considerações Finais}
\label{section:conclusions}

%\subsection{Iniciativas para promover a Sustentabilidade do Software de Pesquisa} \label{subsection:initiatives}

O reconhecimento da necessidade de reprodutibilidade na pesquisa científica, bem como o advento recente da criação de comitês de avaliação de artefatos de pesquisa em conferências e periódicos levaram à recomendação para que os autores  disponibilizem dados de pesquisa, \RS, documentação e materiais usados nas publicações científicas. 
Nesse contexto, o \RSw desenvolvido deve ser sustentável para que cientistas possam entender, replicar e reproduzir pesquisas anteriores ou conduzir novas pesquisas de forma eficaz. 
O suporte à reprodutibilidade ainda demanda que o \RSw seja \textit{FAIR} -- localizável, acessível, interoperável e reutilizável, para responder às necessidades de pesquisa, presentes e futuras. 

%O desenvolvimento de um \RSw requer um ambiente que apoie a sustentabilidade. 
Entretanto, grande parte dos cientistas que escrevem \RSw ainda carecem de uma formação em boas práticas de engenharia de software, por exemplo, uso de sistemas de controle de versão, documentação adequada e testes automatizados.
Neste cenário, a sustentabilidade e outros atributos de qualidade do \RSw podem ficar comprometidos e levar a erros em conclusões científicas e falta de reprodutibilidade na pesquisa que depende do software.

Portanto, ainda que demande tempo e esforço,
uma mudança na forma como o desenvolvimento e a manutenção de \RSw são realizados é necessária, bem como a definição de um modelo de sustentabilidade de software que reconheça a importância de incorporar práticas para promover a sustentabilidade no desenvolvimento de  \RS.
Um caminho natural é investir em treinamento para cientistas de diversas áreas do conhecimento sobre conceitos, boas práticas e ferramentas
para o desenvolvimento de \RSw sustentável.

Ao adotar práticas sustentáveis, os pesquisadores podem criar oportunidades para que outros especialistas na área colaborem no desenvolvimento de seu \RS, forneçam \textit{feedback} e sugestões de melhorias. Além disso, a disseminação dos resultados da pesquisa e a divulgação do software, podem ampliar o impacto além da comunidade acadêmica, beneficiando o público em geral.

%\subsection{Ciência Aberta e Software de Pesquisa no Brasil}
% Ciência Aberta e, em especial, a .
A discussão sobre \RSw sustentável é incipiente e o tema tem sido pouco explorado pela comunidade acadêmica brasileira de Computação. 
O Simpósio Brasileiro de Engenharia de Software (SBES 2022) foi inovador ao apresentar um documento 
delineando Políticas de Ciência Aberta para o evento~\cite{flach:politicas:os}, buscando
encorajar os autores a disponibilizar os artefatos usados na pesquisa -- dados abertos anonimizados e código do \RS, e despertar o interesse na reprodutibilidade dos estudos quantitativos publicados.
O tema do SBES 2022 foi \textit{Sustentabilidade de Software}, e o título de uma das palestras convidadas, proferida pelo Prof. Daniel Katz, foi \textit{Towards Sustainable Research Software}~\cite{katz_daniel_s_2022_7140726}.

\subsection{Institutos e Associações}

Há várias iniciativas que contemplam a questão da \textit{Sustentabilidade do Software de Pesquisa} e 
o apoio a cientistas e engenheiros de software, com o objetivo de promover a adoção de boas práticas no desenvolvimento de \RSw e torná-lo mais sustentável.
%
Alguns países da Europa, América do Norte, Ásia e África têm inaugurado institutos nacionais para apoiar cientistas que desenvolvem \RSw e preparar uma nova categoria de profissionais: os engenheiros de \RS~\cite{jimenez_four_2017}. Entretanto, o Brasil ainda não investiu na criação de organizações ou institutos dedicados a este tema.

Dentre os institutos e associações pioneiros, destacam-se 
(1)~\textit{The Carpentries}\footnote{\url{https://carpentries.org}}, 
(2)~\textit{Software Sustainability Institute} (SSI)\footnote{\url{https://www.software.ac.uk}}, (3)~\textit{Research Software Engineers} (RSE)\footnote{\url{https://researchsoftware.org/}}, 
(4)~\textit{The Society of Research Software Engineering}\footnote{\url{https://society-rse.org}} e  
(5)~\textit{Netherlands eScience Center}~\cite{drost_niels_2020_4020622}. 
%, uma fundação independente baseada na Holanda para desenvolvimento e aplicação de software de pesquisa

\subsection{Cursos e Treinamentos}

Institutos, associações e outras organizações oferecem cursos e treinamentos gratuitos sobre diversos temas ligados ao \RSw e sua sustentabilidade. Recomendamos que visitem os websites e consultem o material disponibilizado.
\begin{itemize}
    \item \textit{The Carpentries}
    é um instituto que oferece treinamento em habilidades computacionais e ciência de dados para cientistas em todo o mundo.
    \item \textit{Software Carpentry} é um projeto voluntário dedicado a ensinar habilidades básicas de computação para pesquisadores~\cite{madicken_munk_2019_3264950, aleksandra_nenadic_2022_6532057}.
    \item \textit{Library Carpentry} é uma comunidade global que ensina habilidades de software e dados para pessoas que trabalham em funções relacionadas a bibliotecas e informações~\cite{madicken_munk_2019_3264950}.
    \item \textit{Netherlands eScience Center} é uma fundação independente baseada na Holanda para desenvolvimento e aplicação de software de pesquisa~\cite{drost_niels_2020_4020622}. 
    \item \textit{The Turing Way} oferece um manual para ciência de dados reprodutível, ética e colaborativa~\cite{the_turing_way_community_2022_7625728}.
\end{itemize}

%\subsection{Modelos de Avaliação}
%O \textit{Software Sustainability Institute}\footnote{\url{https://www.software.ac.uk}}

%A associação de Engenheiros de Software de Pesquisa (\textit{Research Software Engineers}\footnote{\url{https://researchsoftware.org/}} e a \textit{Society of Research Software Engineering}\footnote{\url{https://society-rse.org}})~\cite{katz_addressing_2021}.

%Diretórios de Software de Pesquisa, como o \textit{Research Software Directory}\footnote{https://research-software-directory.org} buscam facilitar a descoberta e reutilização de \textit{Software de Pesquisa}, ajudando a tornar a pesquisa mais transparente e sustentável.
