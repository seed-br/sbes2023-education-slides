%-----------------------------%
\subsection{Reprodutibilidade} \label{subsection:srrs}

A \textit{Reprodutibilidade} é um dos fundamentos do método científico e um princípio norteador da Ciência Aberta~\cite{unesco:2021}.
A boa prática científica exige que os artefatos de pesquisa mencionados em publicações científicas sejam mantidos e fiquem disponíveis para escrutínio dos pares, reprodução independente e verificação de resultados.

%Qual a importância da sustentabilidade do \RS para a pesquisa reprodutível aberta?  
%Lançamentos.
Para o \RS, a submissão ou publicação de um artigo é um dos momentos em que a versão do software usada no estudo precisa ser identificada, documentada e lançada.
% 
Se houver modificações no \RS, elas precisam ser registradas, 
usando algum esquema de nomenclatura que identifique o 
\textit{<software>}, a \textit{<versão>} e o \textit{<lançamento>}.
Em geral, a versão refere-se a mudança estratégica durante a evolução do software e o lançamento refere-se a mudanças simples de serviço.

A recomendação para que pesquisadores compartilhem e permitam o acesso aos dados descritos em suas publicações científicas, não garante a reprodutibilidade da pesquisa. 
Na prática, pesquisadores devem compartilhar o código de todo o \RSw desenvolvido e fluxos de trabalho de suas pesquisas para assegurar a reprodutibilidade.
Na maioria dos casos, o arquivamento de baixo custo do software, acompanhado de documentação clara deve ser suficiente.

Diretrizes para o desenvolvimento sustentável de \RSw também podem contribuir para a condução de uma pesquisa científica reprodutível,
apresentando princípios e boas práticas da engenharia de software.
As implementações das diretrizes podem variar entre os diferentes domínios científicos, mas as implementações devem ser públicas, abertas para discussão entre os pesquisadores e continuamente adaptadas em reposta às mudanças tecnológicas e nos domínios.

%In most cases, cheap software archiving with clear documentation (including version control, etc.) should be sufficient. 
% For this aspect of software sustainability, it is urgently needed that guidelines are given to all researchers, in particular those that are still in a phase before the actual conception of new software, on coding ethics, good practices, and other guidelines to make later use easier, once the software or tools have been created. 
