%------------------------------------------------%
\section{Introdução}
\label{section:introduction}

Nas últimas décadas, produtos de \textit{software} têm assumido um papel fundamental no âmbito da pesquisa científica.
%
Na Ciência Aberta~\cite{unesco:2021}, ao lado dos dados de pesquisa, o software desempenha um papel central, seja como parte do método científico ou como um de seus resultados~\cite{training:handbook}.

Coletar e analisar dados, construir e testar modelos tornaram-se atividades complexas em quase todas as áreas de pesquisa, das ciências exatas às ciências humanas.
%
\cite{hettrick2014uk} relataram que 92\% dos cientistas do Reino Unido usavam software em suas pesquisas, 69\% declararam que a pesquisa não poderia ser realizada sem software, e 56\% desenvolviam seu próprio software, sendo que 21\% destes não possuíam treinamento em desenvolvimento de software.
%
A capacidade de lidar com tal complexidade depende de software especializado, chamado de \textit{software científico}~\cite{hannay:2009}, \textit{software acadêmico} \cite{howison2011scientific} ou \textit{software de pesquisa}.

O termo \textit{Software de Pesquisa}\footnote{Tradução nossa para \textit{Research Software}. O termo \textit{Software de Pesquisa} será utilizado ao longo do texto, sempre em itálico.} foi criado para designar, de modo abrangente, o software utilizado durante a pesquisa científica, incluindo o software de terceiros usado para coleta, processamento e análise de dados~\cite{allen_et_al:DM:2017:7146}.
Recentemente, o uso do termo foi restrito para designar apenas \textit{o software desenvolvido especificamente no contexto da pesquisa}~\cite{nieuwpoort_defining_2023},
com recomendação para que, ao lado dos conjuntos de dados da pesquisa, o código-fonte ou o ambiente de execução do \RSw também sejam disponibilizados.
%
Em tal contexto, há uma preocupação legítima por parte da comunidade científica com a qualidade do \RSw, em especial, a sua sustentabilidade~\cite{carver:rs:2022}. 
A falta de sustentabilidade do \RSw pode ferir a reprodutibilidade na pesquisa, ou a capacidade de reprodução de estudos científicos por pesquisadores independentes, podendo ocasionar graves erros em conclusões centrais da Ciência~\cite{merali2010computational}.

O conceito de \textit{sustentabilidade de software} está relacionado às consequências de longo prazo de projetar, construir e entregar um projeto de software \cite{venters_2014,venters_software_2021}.
Para o \RS, sustentabilidade diz respeito à capacidade do software de perdurar e de continuar sendo suportado ao longo do tempo, o que implica em qualidades de longevidade e manutenibilidade.
O \RSw sustentável deve permanecer utilizável por um longo período de tempo e retornar resultados consistentes (mesmo diante de software e hardware em evolução), com benefícios para a comunidade de pesquisa.
O \RSw sustentável precisa ser atualizado, adaptado para novos ambientes e plataformas e testado.

\cite{carver:rs:2022} realizaram uma pesquisa com 1.149 pesquisadores para identificar os desafios relacionados a sustentabilidade enfrentados no desenvolvimento e uso de software de pesquisa. 
Dentre os resultados, a pesquisa identificou que (i)~o \RSw é desenvolvido e gerenciado sem considerar sua sustentabilidade no longo prazo e, 
(ii)~há necessidade de treinamento para os desenvolvedores de \RSw sobre o ciclo de vida do software e ferramentas para seu desenvolvimento e manutenção.
De fato, grande parte dos cientistas que escrevem \RSw ainda carecem de uma formação em boas práticas de engenharia de software voltadas para o desenvolvimento de software sustentável, por exemplo, uso de sistemas de controle de versão, documentação adequada e testes automatizados.
Neste cenário, a sustentabilidade do \RSw pode ficar comprometida, levar a erros em conclusões científicas e à falta de reprodutibilidade na pesquisa.

Além das preocupações com sustentabilidade, e considerando que software é um artefato de pesquisa digital, o princípio da reprodutibilidade requer que, assim como os dados de pesquisa~\cite{Wilkinson2016}, o \RSw seja \textit{FAIR}, isto é, facilmente localizável, acessível, interoperável e reutilizável~\cite{chue_hong_fair_2022}.
Finalmente, é importante considerar que a escolha de um software ou de uma versão do mesmo software pode ter influência nos resultados da pesquisa: 
as versões do software podem ter funcionalidades parecidas mas com diferenças não documentadas que podem levar as resultados diferentes.

Este capítulo apresenta uma introdução ao \RS, contextualizado no universo da Ciência Aberta, e caracterizado com base em
princípios científicos e práticas da engenharia de software, 
buscando destacar o seu papel na promoção da pesquisa sustentável e reprodutível.

% O principal objetivo deste capítulo é ajudar a esclarecer esses conceitos e explicar a sua importância no contexto da Ciência Aberta.

As definições apresentadas sobre Ciência Aberta têm como base 
a Recomendação da UNESCO sobre Ciência Aberta~\cite{unesco:2021} 
e o Manual de Treinamento em Ciência Aberta~\cite{training:handbook}, disponibilizado sob a licença Creative Commons CC0 1.0 Universal (CC0 1.0) Dedicação ao Domínio Público\footnote{\url{https://creativecommons.org/publicdomain/zero/1.0/}}. 
%
As práticas para o desenvolvimento de \RSw sustentável apresentadas têm como base parte do material público disponibilizado por
\textit{Software Carpentry}~\cite{madicken_munk_2019_3264950, aleksandra_nenadic_2022_6532057}, 
\textit{Library Carpentry}~\cite{madicken_munk_2019_3264950}, 
\textit{Netherlands eScience Center}~\cite{drost_niels_2020_4020622} e 
\textit{The Turing Way}~\cite{the_turing_way_community_2022_7625728}.
%
Finalmente, os exemplos e o material de um estudo realizado com um grupo de pesquisa na área de Física foram criados pelos autores especialmente para este curso e estão disponíveis sob a licença 
licença Creative Commons CC0 1.0 Universal (CC0 1.0)
Dedicação ao Domínio Público.

% Write programs for people, not computers.

