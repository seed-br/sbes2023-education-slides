\section{Exemplo: ferramenta de seleção de projetos}


\url{https://github.com/seed-br/seed.select.git}

Descrever.
Para que o software? 
O problema de seleção de OSS. explicar. 
É resultado da pesquisa: o software é uma contribuição.

Ao mesmo tempo, o software deverá ser avaliado por terceiros, porque foi submetido como tool paper. Explicar.

Ao mesmo tempo, o software deverá ser usado em outras pesquisas, porque ele implementa uma funcionalidade necessária (o software é meio, instrumento).
Por exemplo, ele foi usado em um relato de experiência de aula de ES em que o projeto foi escolhido com o apoio do software.

\begin{enumerate}

    \item Permitir que os usuários filtrem projetos OSS de acordo com até 10 critérios de seleção (tamanho do projeto, linguagem de programação, entre outros).
    
    \item Permitir que usuários autenticados comentem sobre determinado projeto, possibilitando assim anotar possíveis experiências com aquele projeto. Qualquer usuário pode visualizar esses comentários.
    
    \item Allow authenticated users to evaluate a project with 1 to 5 stars to indicate the level of satisfaction in using that particular project. The average rating for each project is available for any user to view.
    
    \item Allow users to select multiple projects and download their metadata in \textit{.csv} format. The metadata includes the project name, URL, number of lines of code, and description.
    
    \item Allow users to visualize the catalog of selection criteria online. 

    \item Allow users to visualize a list with the 10 top contributors of the OSS project.
\end{enumerate}


Inicialmente, o projeto estava em um folder/pasta.
Backup manual. Acesso de apenas uma pessoa.
Primeira funcionalidade OK.

- O software precisa ser hospedado adequadamente.
O software foi colocado no github, repositório privado, inicialmente para permitir que dois desenvolvedores trabalhassem. Ganha controle de versão, backup, etc. 

Não usa issue tracker.

- O software precisa ser confiável, definir e seguir padrões.
Qualidade do software, práticas e ferramentas de suporte.

Não há testes nem foram usadas práticas para garantia da qualidade.


Um estudo foi realizado com a versão ... ??
- O software vai ser divulgado.
Qual a informação que deve estar disponível? README, licença, etc. OK

- Estudo indicou pontos de melhoria.
Nova versão do software.
Versão indicada? Não. Release? Não.

Projeto ain

- Novo estudo com o software. Qual a versão?

%----------------------------------%

- O software precisa ser gerenciado. Quem faz o que? O que está pronto, em andamento? Onde acho essas informações.

- O software precisa ser confiável.
Qualidade do software, práticas e ferramentas de análise estática.



- O software vai ser usado / avaliado.

- V1 foi usada em Paper 1

- O software evolui. Precisa de que? E o próximo paper, qual a versão que usará? E os externos?

- Os externos querem contribuir. O que é necessário?
Code of conduct. Licenças novamente.

