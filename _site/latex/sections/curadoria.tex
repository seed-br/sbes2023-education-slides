\subsection{Curadoria de Software}

Curadoria de software engloba ``as práticas ativas relacionadas à criação, aquisição, avaliação e seleção, descrição, transformação, preservação, armazenamento e disseminação/acesso/reutilização de software em curtos e longos períodos de tempo.'' 
 (Chassanoff, Building a Model for Software Curation).

 %“Software curation encompasses the active practices related to the creation, acquisition, appraisal and selection, description, transformation, preservation, storage, and dissemination/access/reuse of software over short- and long- periods of time.” (Chassanoff, Building a Model for Software Curation).

 In sustaining software one should distinguish aspects of archiving and/or accessibility on the one hand from preservation, maintenance and/or regularly updating on the other, which requires more expertise and an infrastructure to deal with it. 
 
 Software sustainability is significantly different from basic archiving or preservation. 
 Preserving software in principle entails that the code is stored in a trustworthy place that it is accessibly archived, documented (described by metadata) and can be found by search engines. 
 Archiving does not guarantee that the software will still run at a later point in time.
%Sobre Archiving citar Software Heritage
Archiving software is like deep-freezing or canning food so that it can be used some time in the future. In this case, the labels on the can or package function as metadata.



