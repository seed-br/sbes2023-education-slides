\subsection{\textit{FAIRness}}
\label{subsection:fair:software}

%Computational research should be FAIR: Findable, Accessible, Interoperable, Reusable. [CITAR algo]
% A modified version of these principles can be usefully applied to software too:  it should be Findable, Accessible, Reusable and Extensible. [Software as Output]

Os princípios FAIR~\cite{Wilkinson2016} foram especificados para melhorar o reuso de dados de pesquisa digitais, tornando-os mais fáceis de encontrar, acessíveis, interoperáveis e reutilizáveis (\textbf{F}indable, \textbf{A}ccessible, \textbf{I}nteroperable, \textbf{R}eusable).
%
Tais princípios abordam a \textit{forma} de fornecer artefatos para a comunidade científica, mas não tratam do conteúdo funcional ou da qualidade dos artefatos~\cite{lamprecht:2020}.
%\textbf{F}indability, \textbf{A}ccessibility, \textbf{I}nteroperability, \textbf{R}eusability.
Os \textit{Princípios FAIR para Dados de Pesquisa}~\cite{Wilkinson2016}, 
incluindo quatro princípios fundamentais e 15 princípios norteadores, estabelecem que os dados de pesquisa devem ser facilmente localizáveis, acessíveis, interoperáveis e reutilizáveis. 

Segundo~\cite{chue_hong_fair_2022}, o \RSw deve seguir os princípios FAIR usados para dados abertos, considerando-se que
o software também é um artefato de pesquisa digital e, como tal, deve ser facilmente localizável, acessível, interoperável e reutilizável. 
%
Os \textit{Princípios FAIR para Software de Pesquisa} (FAIR4RS) foram definidos a partir de uma reformulação dos princípios FAIR originais para dados abertos~\cite{lamprecht:2020,chue_hong_fair_2022,barker:2022}.
É importante destacar que, diferentemente dos dados, o software não é um artefato estático e só pode ser (re)utilizado se for sustentável~\cite{lamprecht:2020}.
%
A Tabela~\ref{tab:fairness:4:rs:r} apresenta a versão mais recente dos princípios FAIR para \RS.

\textit{Projetos de Software Livre} seguem os princípios FAIR. %recomendados para \RS.
Software livre pode ser localizado em repositórios com base em identificadores e descritores, utilizando diversos critérios como palavras-chave, linguagem de programação, versão do software, entre outros. 
A acessibilidade é encorajada em software disponível em repositórios abertos, com licenças de compartilhamento explícitas e bem definidas e documentação associada. 
A definição de interfaces de programação, formatos de entrada/saída e uso de padrões promovem a interoperabilidade e o reúso por vários grupos de pesquisa.
Nessa perspectiva, as práticas usadas no modelo de desenvolvimento de software livre podem ser adotadas no desenvolvimento e evolução de \RS~\cite{flach:sbc:2021}. 
%\footnote{\url{https://www.sbc.org.br/component/flippingbook/book/53/1?page=1}}

% Fairness template
\begin{table}[htbp]
    \caption{Princípios FAIR para Software~\cite{barker:2022}.}
    \centering
    \small
    \begin{tabular}{p{1.1cm}|p{13cm}}
    \hline
    \textbf{Princ.} & \textbf{Descrição} \\ \hline
    F: & Software, and its associated metadata, is easy for both humans and machines to find.\\
    F1 & Software is assigned a globally unique and persistent identifier.\\
    F1.1 & Software components representing granularity levels are assigned distinct identifiers.\\
    F1.2 & Different versions of the software are assigned distinct identifiers.\\
    F2 & Software is described with rich metadata. \\
    F3 & Metadata clearly and explicitly include the identifier of the software they describe.\\
    F4 & Metadata are FAIR, searchable and indexable.\\ \hline
    A: & Software, and its metadata, is retrievable via standardised protocols.\\
    A1 & Software is retrievable by its identifier using a standardised communications protocol. \\
    A1.1 & The protocol is open, free, and universally implementable.\\
    A1.2 & The protocol allows for authentication and authorization procedure, where necessary. \\
    A2 & Metadata are accessible, even when the software is no longer available.\\ \hline
    %\textbf{Princ.} & \textbf{Descrição} \\ \hline
    I: & Software interoperates with other software by exchanging data and/or metadata, and/or through interaction via application programming interfaces (APIs), described through standards.\\
    I1 &  Software reads, writes and exchanges data in a way that meets domain-relevant community standards.\\
    I2 & Software includes qualified references to other objects.\\
      \hline
    R: & Software is both usable (can be executed) and reusable (can be understood, modified, built upon, or incorporated into other software).\\
    R1 & Software is described with a plurality of accurate and relevant attributes.\\
    R1.1 & Software is given a clear and accessible license. \\ 
    R1.2 & Software is associated with detailed provenance.\\
    R2 & Software includes qualified references to other software.\\
    R3 & Software meets domain-relevant community standards. \\
      \hline
    \end{tabular} \label{tab:fairness:4:rs:r}
\end{table}

%Lots of work beyond FAIR: quality, correctness, reproducibility, openness, ...

%---------------------------------------------%
%Possivel referencia para avaliar se inclui ou nao: https://www.youtube.com/watch?v=67Uc1EEVDv8&t=953s

%O uso de \textit{Software de Pesquisa} é mencionado na literatura por meio de citação formal ou informal~ \cite{smith2016software} e está estreitamente relacionado ao sistema econômico de reputação científica, uma vez que tais menções causam impacto científico direto tanto na publicação quanto no ecossistema de software de pesquisa \cite{katz2014transitive}.

%Conhecimento novo é claramente construído a partir do conhecimento passado e o sistema de citações formais tem promovido avanços significativos \cite{katz2014transitive}.
% No entanto, isso não tem funcionado tão bem para produtos digitais como o software que, muitas vezes, dependem de outro software, fragmentos de código, e algoritmos \cite{katz2014transitive}.


\begin{comment}
\begin{table}[btp]
\begin{tcolorbox}[colback=white,title=Princípios FAIR para Software]
  \begin{description}
    \item \textbf{Localizável} \textit{(\textbf{F}indable)}
    
        O Software possui um rico conjunto de metadados e um identificador único e persistente que facilita sua busca e identificação.
    \item \textbf{Acessível} \textit{(\textbf{A}ccessible)}
    
        Os metadados do Software estão organizados e especificados em um formato legível para pessoas e máquinas. O Software e metadados devem estar depositados em repositórios públicos e reconhecidos pela comunidade.
    \item \textbf{Interoperável} \textit{(\textbf{I}nteroperable)}
    
        O Software usa padrões e plataformas reconhecidos pela comunidade possibilitando a integração com outras ferramentas e sistemas.
    \item \textbf{Reutilizável} \textit{(\textbf{R}eusable)}
    
        O Software possui uma licença e documentação que permitem sua adoção e extensão por outros pesquisadores e desenvolvedores. 
    \end{description}
\end{tcolorbox}
\end{table}
\end{comment}