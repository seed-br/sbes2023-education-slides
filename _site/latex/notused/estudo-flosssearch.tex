%-----------------------------------------------%
\subsection{Avaliação do Software de Pesquisa \texttt{FlossSearch.edu} }
\label{section:casesstudy:flosssearch}

% EXEMPLO PARA GUIAR A APRESENTACAO DAS PRATICAS
%---------------------------------------------------------% 
\begin{table}[htb]
\begin{tcolorbox}[colback=white,title=Nome do Software]
\begin{description}
    \item [FlossSearch.edu.] Software para apoio à busca e seleção de um projeto OSS para uso na Educação em Engenharia de Software.
    \item[Funcionalidades.] xxx.
    \item[Linguagem de programação.] PHP 5.6 (99.3\%)
    \item[Citação.] CCC.
    \item[Versão.] 1.0
    \item[Licença.] GPL
    \item[URL.] ...
\end{description}
\end{tcolorbox}
\end{table}

\subsection*{Sustentabilidade}

\subsection*{Seu website e documentação fornecem uma visão geral clara e de alto nível de seu software?}

As perguntas fundamentais que serão feitas sobre o seu software são: o que ele faz, o que o torna melhor do que outro software que desempenha um papel semelhante e como ele contribui para a pesquisa?
Os usuários em potencial devem ser capazes de encontrar facilmente uma descrição de uma ou duas frases de seu software em seu site e em sua documentação.
Pode ser difícil resumir seu software de forma rápida e concisa, especialmente quando você passou meses codificando centenas de recursos interessantes e poderosos. No entanto, a menos que você chame a atenção de um usuário em potencial muito rapidamente, corre o risco de perdê-lo como um usuário real.

\subsection*{Seu website e documentação descrevem claramente o tipo de usuário que deve usar seu software?}
Se o seu software foi projetado apenas para usuários com tipos específicos de conhecimento ou experiência (relacionados ao domínio de pesquisa e relacionados ao software), você deve ser franco sobre isso em sua documentação. Caso contrário, você corre o risco de atrair usuários que acharão seu software confuso ou inadequado para suas necessidades e que podem reclamar desse fato ou solicitar mais suporte do que você deseja fornecer.

\subsection*{Seu software está disponível como um pacote que pode ser usado ou implantado sem compilar?}
Construir software pode ser complicado e demorado. Fornecer seu software como um pacote que pode ser implantado sem compilar pode economizar tempo e esforço dos usuários para fazer isso sozinhos. Isso pode ser especialmente valioso se seus usuários não forem desenvolvedores de software.
Você deve testar se seu software é compilado e executado em todas as plataformas para as quais ele deve oferecer suporte, o que significa que você já terá criado pacotes que podem ser distribuídos para seus usuários.
Consulte nosso guia \textit{Pronto para Llançamento}? Uma lista de verificação para desenvolvedores (http://www.software.ac.uk/resources/guides/ready-release).
Se você estiver interessado nas consequências de ignorar as necessidades de seus usuários, consulte nosso guia sobre Como frustrar seus usuários, irritar outros desenvolvedores e agradar advogados (http://www.software.ac.uk/resources/guides/how- frustrar-seus-usuários-irritar-outros-desenvolvedores-e-agradar-advogados).

A Tabela~\ref{tab:fairness:moara} ... mostra ... de FlossSearch.
% A Tabela só precisa mostrar o que foi analisado para flossSearch.



\begin{table}[htb]
    \caption{Avaliação de FAIRness - Software de Moara.}
    \centering
    \small
    \begin{tabular}{p{0.9cm}|p{5cm}|c|p{6cm}}
    \hline
    Princ. & Descrição & Atende? & Comentário 
    \\ \hline
     F1 & Software and its associated metadata have a global, unique and persistent identifier for each released version. & SIM & Parcialmente: \textit{Identifier is ... plus version in X.x in all metadata sources. Sources: ... GitHub. It complies with this principle from version 2.0. It does not have a specific PID, but it can be easily found across different repositories and registries including version information.} \\
     F2 & Software is described with rich metadata. & & \\
     .. & .. & .. & ..\\
     A1 & Software and its associated metadata are accessible by their identifier using a standardized communications protocol. & YES & \textit{Both software and metadata are accessible through HTTPS ..: [..].}
     \\
    .. & .. & .. & ..\\
     R1.1 & Software and its associated metadata have independent, clear and accessible usage licenses compatible with the software dependencies. & YES & 
     \textit{Software: GNU General Public License as published by the Free Software Foundation, either version 3 of the license or later versions.}
     \\
      .. & .. & .. & ..\\
      \hline
    \end{tabular} \label{tab:fairness:moara}
\end{table}


